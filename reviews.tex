

------------------------
Reviewer #1: 

(1) While I enjoyed reading Section 2 and 3, they seem to be not strongly related to the main results (Section 4 / 5). It was confusing to me (someone not familiar with this topic) what the main points of the paper are, till I finished reading Section 4 and 5. I think it would have been better if the paper can go directly to its  main results and then give some examples afterwards. The relation between Section 3 to Sections 4/5 can also be made more clearly.

(2) Theorem 4.1 provides sufficient conditions for ``good observation functions". It would be good to give some discussions on how ``tight'' these conditions are (how far they are from being necessary). In the original Takens' embedding theorem (Theorem 1.1), the embedding dimension is $2m+1$ (where $m$ is the dimension of the hidden manifold $M$). Theorem 4.1 does not provide bounds on its $N$ and $\tau$. Is it hopeful to bound them at all in general? Could stronger conditions lead to bounds on them too?

(3) The practical implications of results in this paper are not very clear to me. The authors should discuss that.




Reviewer #2

The paper is concerned with Takens’ delay embedding theorem, which asserts
that the phase space of a dynamical system can be ‘reconstructed’ by using
sliding window of a generic observation along trajectories. The main theoretical
result of the paper, Theorem 4.1, gives an explicit, sufficient condition for an
observation to yield a Takens embedding embedding. The second objective
of the paper is to combine Takens embedding with peristent homology and
Eilenberg-MacLane coordinates, in order to recover loops high-dimensional tori
(quasi-periodic trajectories), Klein bottles, sphere and projective planes from
time series. Quite a few examples and numerical experiments illustrate the
theory.
The paper is novel and relevant to the field, and deserves publication.
Comments and questions:
– Is the condition in Theorem 4.1 also necessary for an embedding?
– The stability of the geometric objects of interest that are described by
trajectories – e.g. loops, tori – may play a role in practice in the reconstruction
of these objects from the time series; for instance, one may be able to reconstruct
a stable torus and may be unable to reconstruct an unstable one. In the case
of the torus, the arithmetic properties of the frequency vector for the irrational
flow may play more complex role in reconstructing the dynamics; e.g., in the
context of conservative dynamics one desirable condition is that the frequency
vector satisfies a Diophantine condition in order to guarantee stability. It would
be interesting to see of the authors have any comments on that.





Reviewer #3:

Summary/Contributions:

Taken's embedding theorem is hugely important in dynamical systems, allowing for the reconstruction of the state space of a dynamical system. However, Takens embeddings rely on a good choice of a good observation function, but does not provide clear conditions for such a function. This has proven to be problematic in practice. The main contribution is to define "good" and provide conditions on determining good observation functions with particular dynamics. This provides a theoretical framework for extending Takens' embedding theorem.

In short, the pipeline is as follows: One starts with a manifold, define a vector field and a flow  for which one chooses a good observation function (chosen to meet criteria given in the paper).  Then, using a sliding window (with correctly chosen parameters) the manifold is reconstructed. Presumably, in applications start with a flow (signal), choosing a good observation function allows for the manifold to be reconstructed. Other tools may be applied to detect topological structure (persistent homology) or for dimension reduction.

They begin with a family of distance based observation function and extend to other families of functions. They proceed to give conditions (theorems and proofs)  for good observation functions for  the torus and the Klein bottle.

This paper provides a clear method for generating interesting examples. Having these types of examples is vital for the comparison of signals in "the wild" to known systems. Being able to identify the state space is important in application.  Already, the few example known (ie the sliding window embedding of a periodic signal produces a loop and a quasiperiodic signal, a tori) have been proven to be useful in detecting structure in application. This paper promises to extend such results to more interesting manifolds.

Further, they demonstrate how tools such as Eilenberg-MacLane coordinates can be used as a nonlinear dimension reduction step. Here they employ this for validation of results, but it is easy to see how this could be applied in other contexts.


Assessment

Overall this paper is well written. The authors demonstrated through well chosen examples why certain conditions on the theorems are needed motivating examples and provide rigorous proofs of the main theoretical results. They expand the main result with several other theorems, presented with more motivating/ illuminating examples. Overall this extends the powerful results from Takens theorem  and allows for the generation of families of examples of other manifolds, significantly extending the current suite of well-understood examples.

An initial concern might be with the fit of the journal - the main result leans in the direction of differential topology or differential geometry. However, their main results are of interest to the audience of this journal, which includes practitioners working on applying topological techniques to signals. In fact this is a growing direction for the applied topology community and this paper provides a rigorous framework.





While not necessary, this paper could be strengthened by the addition of a compelling example where this technique is used, or at least expand the conclusions. The authors mention this briefly at the end. Even the case for this could be made stronger (maybe even highlighting the adjusted vector field in 5.2 a little more).



Suggestions:

It would be helpful if the authors filled out the conclusion a little more clearly. I think these results are  useful and important, but the point could be emphasized at the end.

page 22 line 7 : " It is unlikely that one could identify said time series in the wild without such examples in hand"  Naively, one might think "is it really that hard to choose a good function?" Can you briefly comment as to why this is indeed hard.

It is not entirely clear to me that the irrational windings offer a broad class for signals. Can you comment a little on this or  the possibility of including other dynamics?

More of a curiosity, is the condition that points are evenly sampled easy to achieve in practice/application?

There are several minor things to address:

    1. Is there a specific reason the authors break the convention of alphabetical order for the list of authors? If there is, possibly consider including contributions in the body of the paper instead.
    4. Fig 2 - relabel stereo 1 and 2 to match points labeled in persistence diagram   <- Chris
    5. ex 3.1 - how were window sizes chosen - briefly mention heuristic?
    8. Ex 3.1 - could you mention briefly the intuition behind using H_1 instead of H_2 for the coordinate map? Do you just choose the circle map for ease?
    11. Is an irrational winding reasonable to expect or easy to construct on an unknown manifold?
